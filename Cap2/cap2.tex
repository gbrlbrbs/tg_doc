
This chapter will introduce Machine Learning (ML) concepts and techniques being explored in this work, namely the classification problem, neural networks and the Generative Adversarial Network architecture.

\section{Classification}

In Statistics and Machine Learning, a problem is defined as a classification problem when it consists in identifying which categories a population belongs to. An example might be identifying which race of domestic cat is shown in a picture containing a cat. An algorithm that implements classification is known as a classifier. The classifier works by analysing each observation into dependent variables and either mapping those to the categories or by comparing each observation to previous observations by means of a similarity function or loss function. 

Terminology between Statistics and Machine Learning tend to differ. In this work, we will be using the terminology found in Machine Learning, namely:

\begin{itemize}
	\item dependent variables are called features;
	\item categories are called classes;
\end{itemize}

There are many algorithms for classification. 

\section{Neural networks}



\section{Generative Adversarial Networks}